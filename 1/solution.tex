 
\documentclass[11pt]{article}

\usepackage[slovak]{babel}
\usepackage[utf8]{inputenc}
\usepackage[T1]{fontenc}
\usepackage{graphicx}

\usepackage{times}


\usepackage[margin=1in]{geometry} 
\usepackage{amsmath,amsthm,amssymb}
 
\newcommand{\N}{\mathbb{N}}
\newcommand{\Z}{\mathbb{Z}}
 
\newenvironment{theorem}[2][Theorem]{\begin{trivlist}
\item[\hskip \labelsep {\bfseries #1}\hskip \labelsep {\bfseries #2.}]}{\end{trivlist}}
\newenvironment{lemma}[2][Lemma]{\begin{trivlist}
\item[\hskip \labelsep {\bfseries #1}\hskip \labelsep {\bfseries #2.}]}{\end{trivlist}}
\newenvironment{exercise}[2][Exercise]{\begin{trivlist}
\item[\hskip \labelsep {\bfseries #1}\hskip \labelsep {\bfseries #2.}]}{\end{trivlist}}
\newenvironment{reflection}[2][Reflection]{\begin{trivlist}
\item[\hskip \labelsep {\bfseries #1}\hskip \labelsep {\bfseries #2.}]}{\end{trivlist}}
\newenvironment{proposition}[2][Proposition]{\begin{trivlist}
\item[\hskip \labelsep {\bfseries #1}\hskip \labelsep {\bfseries #2.}]}{\end{trivlist}}
\newenvironment{corollary}[2][Corollary]{\begin{trivlist}
\item[\hskip \labelsep {\bfseries #1}\hskip \labelsep {\bfseries #2.}]}{\end{trivlist}}
 
\begin{document}
 
% --------------------------------------------------------------
%                         Start here
% --------------------------------------------------------------
 
%\renewcommand{\qedsymbol}{\filledbox}
 
\title{TIN - Domáca úloha č. 1}%replace X with the appropriate number
\author{Roman Dobiáš - xdobia11}
 
\maketitle

\section*{Úloha č.1}
\begin{enumerate}
\item 
    %%% 1 a) 
    Nech $L_1, L_2 \in \mathcal{L}_3$. 
    Pomocou axiomu doplnku z teórie množín je možné previesť nasledujúcu úpravu:
        \begin{align*}
            L_1 \setminus L_2 = L_1 \cap \overline{L_2}
        \end{align*}
    Z vety 3.23 (skripta, str. 50) vieme, že trieda jazykov $\mathcal{L}_3$ je uzavretá voči
    $\cup$ a $\cap$ voči triede $\mathcal{L}_3$, a doplnku ku $\Sigma^{*}$.
            Nakoľko doplnok aj prienik sú uzavreté operácie v $\mathcal{L}_3$, potom $L_1 \cap
    \overline{L_2} \in \mathcal{L}_3$,a vzhľadom na ekvivalenciu potom aj $L_1 \setminus L_2 \in
    \mathcal{L}_3$. 
\item 
    %%% 1 b)    
    Nech $L_1 \in \mathcal{L}_3, L_2 \in \mathcal{L}_2^D$. Z vety 4.27 vieme, že trieda
        $\mathcal{L}_2^D$ je uzavretá voči prieniku s jazykom triedy $\mathcal{L}_3^D$ a voči
        doplnku. 
        Preto $\overline{L_2} \in \mathcal{L}_2^D$ a zrejme $L_1 \cap \overline{L_2} \in
        \mathcal{L}_2^D$. Zároveň vieme, že $L_1$ aj $L_2$ sú množiny a platí pre ne vzťah $L_1
        \setminus L_2 = L_1\cap \overline{L_2}$. Preto nutne $L_1\setminus L_2 \in L_2^D$.

\item 
    %%% 1 c)    
    Z vety 4.24 plynie, že trieda jazykov $L_2$ nie je uzavretá voči prieniku, ani doplnku. Preto
        $L_1 \cap \overline{L_2} \notin \mathcal{L}_2$, zároveň platí $L_1
        \setminus L_2 = L_1\cap \overline{L_2}$ a preto $L_1 \setminus L_2 \notin
        \mathcal{L}_2$.
\end{enumerate}
\section*{Úloha č.2}
    Nech $P = (\{q_0, q_1, q_2, q_3\}, \{0,1,2\}, \{a,Z\}, \delta, q_0, Z, \{q_3\})$ je deterministický zásobníkový automat, kde
    $\delta$ je definované nasledovne:
    \begin{align*}
        \delta (q_0, 1, Z) &= (q_0, aZ)\\
        \delta (q_0, 1, a) &= (q_0, aa)\\
        \delta (q_0, 2, Z) &= (q_0, aaZ)\\
        \delta (q_0, 2, a) &= (q_0, aaa)\\
        \delta (q_0, 0, Z) &= (q_0, Z)\\
        \delta (q_0, 0, a) &= (q_0, a)\\
        \delta (q_0, c, a) &= (q1, a)\\
        \delta (q_0, c, Z) &= (q1, Z)\\
        \delta (q1, a, a) &= (q1, \epsilon)\\
        \delta (q1, b, a) &= (q2, \epsilon)\\
        \delta (q1, 0, a) &= (q1, a)\\
        \delta (q2, \epsilon, a) &= (q1, \epsilon)\\
        \delta (q1, \epsilon, Z) &= (q3, \epsilon)\\
    \end{align*}


\section*{Úloha č.3}
TO BE DONE

\section*{Úloha č.4}

\textbf{Vstup:} Pravá lineárna gramatika $G_P = (N, \Sigma, P, S)$ taká, že $P$ obsahuje len pravidlá typu
$A\to xB, x \in (N\cup\Sigma)^*, A,B \in N$ a $A\to x, x \in (N\cup\Sigma)^*, A \in N$\\
\textbf{Výstup:} Ľavá lineárna gramatika $G_L = (N_1, \Sigma, P_1, S_1)$ taká, že $P$ obsahuje len pravidlá typu
$A\to Bx, x \in (N\cup\Sigma)^*, A,B \in N$ a $A\to x, x \in (N\cup\Sigma)^*, A \in N$\\
\textbf{Metóda:}
\begin{enumerate}
    \item $N_1 = N \cup \{S_1\}$
    \item Pre každé pravidlo z množiny $P$ typu $A\to xB, A,B \in N, x \in (N\cup\Sigma)^*$:\\
        \begin{align*}
            (B,Ax) \in P_1 \iff (A, xB) \in P
        \end{align*}
    \item Pre každé pravidlo z množiny $P$ typu $A\to x, A \in N, x \in (N\cup\Sigma)^*$:\\
        \begin{align*}
            (S_1,Ax) \in P_1 \iff (A, x) \in P
        \end{align*}
    \item $P_1 = P_1 \cup \{S\to\epsilon\}$
\end{enumerate}

\subsection*{Demonstrace}
\begin{enumerate}
    \item $N_1 = \{S, A, B, S_1\}$
    \item $P_1$: 
            \begin{align*}
                A&\to Sab  & S&\to Sb\\
                B&\to Ab   & B&\to Ab\\
                S&\to A    & A&\to Ba
            \end{align*}
    \item $P_1$: 
            \begin{align*}
                S_1&\to Aab & S_1&\to B\epsilon
            \end{align*}
    \item $G_L = (N_1, \{a,b\}, P_1, S_1)$

\end{enumerate}

\section*{Úloha č.5}
Podľa Myhill-Nerudovej vety pre jazyk $L$ existuje deterministický konečný automat M taký, že $L(M)
= L$, práve vtedy ak prefixová ekvivalencia $\sim_L$ má konečný index. 
 
\end{document}
