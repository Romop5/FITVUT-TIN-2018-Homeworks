 
\documentclass[10pt]{article}

\usepackage[slovak]{babel}
\usepackage[utf8]{inputenc}
\usepackage[T1]{fontenc}
\usepackage{graphicx}

\usepackage{times}


\usepackage[margin=1in]{geometry} 
\usepackage{amsmath,amsthm,amssymb}
 
\newcommand{\N}{\mathbb{N}}
\newcommand{\Z}{\mathbb{Z}}
 
\newenvironment{theorem}[2][Theorem]{\begin{trivlist}
\item[\hskip \labelsep {\bfseries #1}\hskip \labelsep {\bfseries #2.}]}{\end{trivlist}}
\newenvironment{lemma}[2][Lemma]{\begin{trivlist}
\item[\hskip \labelsep {\bfseries #1}\hskip \labelsep {\bfseries #2.}]}{\end{trivlist}}
\newenvironment{exercise}[2][Exercise]{\begin{trivlist}
\item[\hskip \labelsep {\bfseries #1}\hskip \labelsep {\bfseries #2.}]}{\end{trivlist}}
\newenvironment{reflection}[2][Reflection]{\begin{trivlist}
\item[\hskip \labelsep {\bfseries #1}\hskip \labelsep {\bfseries #2.}]}{\end{trivlist}}
\newenvironment{proposition}[2][Proposition]{\begin{trivlist}
\item[\hskip \labelsep {\bfseries #1}\hskip \labelsep {\bfseries #2.}]}{\end{trivlist}}
\newenvironment{corollary}[2][Corollary]{\begin{trivlist}
\item[\hskip \labelsep {\bfseries #1}\hskip \labelsep {\bfseries #2.}]}{\end{trivlist}}
 
\begin{document}
 
% --------------------------------------------------------------
%                         Start here
% --------------------------------------------------------------
 
%\renewcommand{\qedsymbol}{\filledbox}
 
\title{TIN - Domáca úloha č. 2}%replace X with the appropriate number
\author{Roman Dobiáš - xdobia11@stud.fit.vutbr.cz}
 
\maketitle

\section*{Úloha č.1}
Podľa definície 4.29 (opora, str. 97) je Dyckov jazyk na jednou dvojicou [] generovaný 
gramatikou s nasledujúcimi pravidlami, ktorú označme $G_D$:
\begin{align*}
    S \to [ S ] | SS | \epsilon
\end{align*}
Pre túto gramatiku zjavne platí, že každá postupnosť derivácii vedie ku terminálnemu reťazcu.

Uvažujme neprázdny reťazec $w \in L$. Uvažujme prvú deriváciu štartujúceho nonterminálu $S$ v
gramatike $G_D$. Potom podľa definície gramatiky $G_D$ sú 3 možnosti, ktoré pravidlo mohlo byť
použité v prvej derivácii pri generovaní reťazca $w$:
\begin{enumerate}
    \item Pravidlo $S \to \epsilon$\\
        Potom $S \Rightarrow \epsilon$, teda $w = \epsilon$, čo je v spore s predpokladom, že
        $w$ je neprázdny reťazec. Pravidlo teda nemôže byť použité v prvej derivácii.
    \item Pravidlo $S \to [S]$\\
        Potom $S \Rightarrow [S] \Rightarrow^+ w$. Potom je reťazec $w$ nutne konkatenáciou
        tvaru $[u], u \in L$, pretože S je počiatočný nonterminál gramatiky $G_D$, teda derivuje
        slovo $u \in L$. Zároveň platí, že $|w| =  2+|u|$.
    \item Pravidlo $S \to SS$\\
        Vieme, že $w$ je neprázdne. V gramatike $G_D$ obsahuje len jedno pravidlo terminálne symboly,
        preto zjavne pravidlo $S \to [S]$ musí byť použité aspoň raz v postupnosti derivácii vetnej
        formy $SS$. Nutne teda platí, že z vetnej formy $SS$ deriváciami získame vetnú formu $S^N$.
        Zjavne aspoň jedno $S$ musí byť prepísané pravidlom $S \to [S]$. Zaujíma nás ten najľavejší
        prepísaný nonterminál $S$. Potom vzniknutá vetná forma má tvar $[S]S^N$. Zjavne platí, že
        $S \Rightarrow SS \Rightarrow SSS \Rightarrow^* S^N \Rightarrow v, v \in L$. Teda reťazec $w$
        je konkatenáciou $[u]v$, kde $S\Rightarrow^*u$ a $S \Rightarrow* S^N \Rightarrow* v$. Zjavne
        $|w| = 2 + |u|+|v|$.
\end{enumerate}
Ukázali sme teda, že každý neprázdny reťazec $w \in L$ je možné rozložiť na $[u]v, u,v \in L$.

\begin{proof}
    Dokazujeme, že $\forall i \in N: \forall w \in L: \#_[(w) = i \implies w \in L(G)$.\\
    Pre $i = 0$ tvrdenie platí, pretože $\epsilon \in L \land \#_[(\epsilon) = 0) \land \epsilon \in
    L(G).$ \\

    Predpokladajme, že tvrdenie platí pre $j <  i$ a uvažujme indukčnú hypotézu $i = j+1 \land
    \forall w \in L: \#_[(w) = i \implies w \in L(G)$.

    V a) sme ukázali, že pre $\forall w \in L: \exists u,v \in L: w = [u]v$. Teda isto platí, že
    $\forall w \in L \land \#_[(w) = i: \exists u,v \in L: w = [u]v$ a platí, že $\#_[(u) < i$ a zároveň
    $\#_[(v) < i$, pretože konkatenácia $[u]v$ pridáva reťazcu $w$ nutne o jeden symbol $[$ viac.
    Teda nutne $u,v \in L(G)$, pretože pre všetky $w\in L, \#_[(w) < i$ je veta už dokázaná. Zároveň $S \Rightarrow_G [S]S \Rightarrow_G^* [u]v$,
    teda $[u]v \in L(G)$, čiže $w \in L(G)$. \\

    Tvrdenie teda platí pre $i = j + 1$, kde pre $j$ je dokázané, teda platí pre $\forall i \in N$. 
\end{proof}

\end{document}
