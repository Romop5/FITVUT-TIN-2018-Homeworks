 
\documentclass[10pt]{article}

\usepackage[slovak]{babel}
\usepackage[utf8]{inputenc}
\usepackage[T1]{fontenc}
\usepackage{graphicx}

\usepackage{times}


\usepackage[margin=1in]{geometry} 
\usepackage{amsmath,amsthm,amssymb}
 
\newcommand{\N}{\mathbb{N}}
\newcommand{\Z}{\mathbb{Z}}
 
\newenvironment{theorem}[2][Theorem]{\begin{trivlist}
\item[\hskip \labelsep {\bfseries #1}\hskip \labelsep {\bfseries #2.}]}{\end{trivlist}}
\newenvironment{lemma}[2][Lemma]{\begin{trivlist}
\item[\hskip \labelsep {\bfseries #1}\hskip \labelsep {\bfseries #2.}]}{\end{trivlist}}
\newenvironment{exercise}[2][Exercise]{\begin{trivlist}
\item[\hskip \labelsep {\bfseries #1}\hskip \labelsep {\bfseries #2.}]}{\end{trivlist}}
\newenvironment{reflection}[2][Reflection]{\begin{trivlist}
\item[\hskip \labelsep {\bfseries #1}\hskip \labelsep {\bfseries #2.}]}{\end{trivlist}}
\newenvironment{proposition}[2][Proposition]{\begin{trivlist}
\item[\hskip \labelsep {\bfseries #1}\hskip \labelsep {\bfseries #2.}]}{\end{trivlist}}
\newenvironment{corollary}[2][Corollary]{\begin{trivlist}
\item[\hskip \labelsep {\bfseries #1}\hskip \labelsep {\bfseries #2.}]}{\end{trivlist}}
 
\begin{document}
 
% --------------------------------------------------------------
%                         Start here
% --------------------------------------------------------------
 
%\renewcommand{\qedsymbol}{\filledbox}
 
\title{TIN - Domáca úloha č. 2}%replace X with the appropriate number
\author{Roman Dobiáš - xdobia11@stud.fit.vutbr.cz}
 
\maketitle

\section*{Úloha č.1}
Podľa definície 4.29 (opora, str. 97) je Dyckov jazyk na jednou dvojicou [] generovaný 
gramatikou s nasledujúcimi pravidlami, ktorú označme $G_D$:
\begin{align*}
    S \to [ S ] | SS | \epsilon
\end{align*}
Pre túto gramatiku zjavne platí, že každá postupnosť derivácii vedie ku terminálnemu reťazcu.

Uvažujme neprázdny reťazec $w \in L$. Uvažujme prvú deriváciu štartujúceho nonterminálu $S$ v
gramatike $G_D$. Potom podľa definície gramatiky $G_D$ sú 3 možnosti, ktoré pravidlo mohlo byť
použité v prvej derivácii pri generovaní reťazca $w$:
\begin{enumerate}
    \item Pravidlo $S \to \epsilon$\\
        Potom $S \Rightarrow \epsilon$, teda $w = \epsilon$, čo je v spore s predpokladom, že
        $w$ je neprázdny reťazec. Pravidlo teda nemôže byť použité v prvej derivácii.
    \item Pravidlo $S \to [S]$\\
        Potom $S \Rightarrow [S] \Rightarrow^+ w$. Potom je reťazec $w$ nutne konkatenáciou
        tvaru $[u], u \in L$, pretože S je počiatočný nonterminál gramatiky $G_D$, teda derivuje
        slovo $u \in L$. Zároveň platí, že $|w| =  2+|u|$.
    \item Pravidlo $S \to SS$\\
        Vieme, že $w$ je neprázdne. V gramatike $G_D$ obsahuje len jedno pravidlo terminálne symboly,
        preto zjavne pravidlo $S \to [S]$ musí byť použité aspoň raz v postupnosti derivácii vetnej
        formy $SS$. Nutne teda platí, že z vetnej formy $SS$ deriváciami získame vetnú formu $S^N$.
        Zjavne aspoň jedno $S$ musí byť prepísané pravidlom $S \to [S]$. Zaujíma nás ten najľavejší
        prepísaný nonterminál $S$. Potom vzniknutá vetná forma má tvar $[S]S^N$. Zjavne platí, že
        $S \Rightarrow SS \Rightarrow SSS \Rightarrow^* S^N \Rightarrow v, v \in L$. Teda reťazec $w$
        je konkatenáciou $[u]v$, kde $S\Rightarrow^*u$ a $S \Rightarrow* S^N \Rightarrow* v$. Zjavne
        $|w| = 2 + |u|+|v|$.
\end{enumerate}
Ukázali sme teda, že každý neprázdny reťazec $w \in L$ je možné rozložiť na $[u]v, u,v \in L$.

\begin{proof}
    Dokazujeme, že $\forall i \in N: \forall w \in L: \#_[(w) = i \implies w \in L(G)$.\\
    Pre $i = 0$ tvrdenie platí, pretože $\epsilon \in L \land \#_[(\epsilon) = 0) \land \epsilon \in
    L(G).$ \\

    Predpokladajme, že tvrdenie platí pre $j <  i$ a uvažujme indukčnú hypotézu $i = j+1 \land
    \forall w \in L: \#_[(w) = i \implies w \in L(G)$.

    V a) sme ukázali, že pre $\forall w \in L: \exists u,v \in L: w = [u]v$. Teda isto platí, že
    $\forall w \in L \land \#_[(w) = i: \exists u,v \in L: w = [u]v$ a platí, že $\#_[(u) < i$ a zároveň
    $\#_[(v) < i$, pretože konkatenácia $[u]v$ pridáva reťazcu $w$ nutne o jeden symbol $[$ viac.
    Teda nutne $u,v \in L(G)$, pretože pre všetky $w\in L, \#_[(w) < i$ je veta už dokázaná. Zároveň $S \Rightarrow_G [S]S \Rightarrow_G^* [u]v$,
    teda $[u]v \in L(G)$, čiže $w \in L(G)$. \\

    Tvrdenie teda platí pre $i = j + 1$, kde pre $j$ je dokázané, teda platí pre $\forall i \in N$. 
\end{proof}


\section*{Úloha č.2}
Dôkaz predvedieme pomocou Pumping Lemma pre bezkontextové jazyky. Dokážeme, že jazyk nie je
bezkontextový.

\begin{proof}
Nech $L$ je bezkontextový jazyk nad abecedou $\Sigma$. Potom existuje konštanta $k > 0$ taká, že platí:
    \begin{align}
        \forall z \in L \land |z| \geq k: \exists u,v,w,x,y \in \Sigma^*: z = uvwxy \land |vwx| \leq
        k \land \forall i \geq 0: uv^iwx^iy \in L
    \end{align}
Nech existuje konštanta $k > 0$ a reťazec $z=a^p$, kde $p$ je prvočíslo a zároveň $p > n$. Potom
podľa Pumping Lemma platí, že:
    \begin{align}
        \exists u,v,w,x,y \in \Sigma^*: z = uvwxy \land |vwx| \leq
        k \land \forall i \geq 0: uv^iwx^iy \in L
    \end{align}
Uvažujme ľubovoľné konštanty $b, c \in N$ také, že $a^ba^c \neq \epsilon$ a $u = a^b \land v =
    a^c$. Zároveň zvoľme ľubovoľne $uwz$ tak, nech platí, že $z = uvwxy$.
    Plati, že $uvwxy = ua^bwa^cy \in L$.

Uvažujme iteráciu pumpovania $i = 1+p$. 
    Potom $|uv^{i+p}wx^{i+p}y| = |ua^{b*(1+p)}wa^{c*(1+p)}y| = |ua^{b+bp}wa^{c+cp}y| = |ua^bwa^cy| +
    |a^{bp}|+|a^{cp}| = p + bp + cp = p(1+b+c)$. 
    Podľa predpokladu má platiť, že $uv^{i+p}wx^{i+p}y \in L$, avšak dĺžka tohoto reťazca je
    $p(1+b+c)$, čo je súčin prirodzených čísiel, teda nie je prvočíslom, teda $uv^{i+p}wx^{i+p}y \notin L$, čo je spor
    predpokladu. 
    Jazyk ${a^p | p je prvočíslo}$ nie je bezkontextovy.
\end{proof}



\section*{Úloha č.3}
Redukcia
\begin{enumerate}
    \item MP problém $\{<M>\#<w> |$ TS M zastaví na reťazci s kódom w$\}$
    \item Affine problém $\{<M>| $jazyk TS M obsahuje aspoň 1 reťazec z jazyka Affine$\}$
\end{enumerate}
Idea sigmy
\begin{enumerate}
    \item Pre danú inštanciu MP problému Sigma vygeneruje kód TS taký, že:
    \item TS vymaže svoju vstupnú pásku
    \item TS skontroluje zda pôvodný TS <M> je well-formed (nie je garbage), ak nie je well-formed,
        odmietne.
    \item TS skopíruje kód reťazca <w> na svoju pásku
    \item TS pomocou UTS simuluje <M> na kóde <w>
    \item Ak UTS cyklí, potom jazyk TS je prázdny, teda neobsahuje reťazec z Affine (a
        <M>\#<w> nepatrí do MP.)
    \item Ak zastaví, tak TS prijime.
\end{enumerate}
Skúmajme jazyk L(Mx):
\begin{enumerate}
    \item $L(Mx) = \emptyset \iff$ <M> nie je well-formed alebo TS <M> na slove <w> nezastaví
    \item $L(Mx) = \Sigma^* \iff$ <M> je well-formed a <M> zastaví na <w>
\end{enumerate}
Ekvivalencia
\begin{enumerate}
    \item TS M prijme <w> $\iff$ TS M je správne sformovaný a TS <M> na slove <w> zastaví $\iff L(Mx) =
        \Sigma^* \iff$ TS <M> taký, že jazyk L(M) obsahuje aspoň jeden reťazec z jazyka Affine
\end{enumerate}
\section*{Úloha č.4}
\subsection*{A - TS na RationalC}
Prevod TS na RationalC sa skladá z nasledujúcich částí
\begin{enumerate}
    \item Prevod pásky $w$ na $x_0$\\
        Páska $w$ obsahujúca znaky $0,1,\Delta$ je prevedená nasledujúco:
        \begin{enumerate}
            \item 0 je vyjadrená ako 10
            \item 1 je vyjadrená ako 11
            \item $\Delta$ je vyjadrená ako 00
        \end{enumerate}
        Po prevedení symbolov w na takéto binárne kódovanie je výsledný reťazec interpretovaný ako
        binárne zakódované racionálne číslo, ktorého desatinná čiarka je za prvými dvoma binárnymi
        číslami (za prvou dvojicou).
        Príklad w: $\Delta011\Delta$ sa zobrazí na $00. 10 11 11 00$, teda na racionálne číslo
        $0.734375$. Vďaka tomuto zakódovaniu obsahuje racionálne číslo nekonečno znakov
        reprezentujúcich $\Delta$ z prava aj z ľava pôvodonej pásky $w$, pretože dvojica 00 pridaná
        pred aj na koniec binárnej reprezentácie čísla nemení jeho význam.
    \item Prevod jednotlivých pravidiel TS na príkazy jazyka RacionalC
        Jednotlivé stavy TS sú v RationalC reprezentované sekvenciou príkazov. 
        Keďže jednotlivé symboly TS sú v RationalC reprezentované dvojicami bitov racionálneho čísla, potom posun hlavy je
        realizovaný 2x násobením alebo 2x delením, čo má za následok posun desatinnej čiarky o
        dvojicu bitov.
        Pre každý prechod p TS zo stavu q je vygenerovaný príkaz nasledujúco:
\end{enumerate}
        \begin{verbatim}
        q: x /= 2   
        q+1 if x \% 2 == 0 goto q-6 
        q+2 x *= 2
        q+3 if x \% 2 == 0 goto q-0 
        q+4 if x \% 2 == 1 goto q-1 
        q+5 return B;
        q+6 x *= 2
        q+7 if x \% 2 == 0 goto q-delta
        q+8 if x \% 2 == 1 goto q-delta
        \end{verbatim}
        Hodnota $B$ riadku $q+5$ je 0 pre nekoncový stav a $1$ pre $q_f$.
        V prípade, že stav $q$ neobsahuje prechod pre daný symbol z $\Gamma$, je daný riadok
        nahradený prázdnym riadkom (pre $\Delta$ riadok $q+1$, pre $0$ $q+3$, pre $1$ $q+4$.

        Povšimnime si, že ak stav nie je koncový a zároveň neobsahuje v TS žiaden prechod, RationalC
        skončí program s návratovou hodnotou 0, čo je ekvivalentné abnormálnemu zastaveniu TS na
        nedefinovanom prechode.

        Symboly $q-delta, q-0, q-1$ tvaru $Q-\Gamma$ predsavujú symbolický vyjadrenie čísla riadku, ktorý
        reprezentuje sekvenciu príkazov vykonávajúci konkrétny prechod.
        Napríklad prechodu $\delta(q, 0) = (p, L)$ zodpovedá sekvencia na adrese $q-0$:
        \begin{verbatim}
        q-0: x /=2
        x /=2
        if \% 2 == 0 goto p
        if \% 2 == 1 goto p
        \end{verbatim}
        Pre prechod R získame sekvenciu analogicky, nahradením x /=2 za x *=2.
        Pri zápise symbolu z abecedy $\Gamma$ nahradíme dvojicu násobení / delení nasledujúcou
        sekvenciou:
        \begin{enumerate}
            \item Zápis $\Delta$ ako 00: x = even(x); x /= 2; x = even(x); x */ 2;
            \item Zápis $0$ ako 10: x = even(x); x /= 2; x = odd(x); x */ 2;
            \item Zápis $0$ ako 11: x = odd(x); x /= 2; x = odd(x); x */ 2;
        \end{enumerate}
        Príkazy $odd()$ a $even()$ pracujú s bitom racionálneho čísla pred desatinnou čiarkou, teda
        umožňujú simulovať zápis znaku na dvojici bitov, ktorá je pred desatinnou čiarkou, teda
        odpovedá pôvodnemu symbolu pod hlavou v TS.

        Ošetrenie prepadnutia hlavy:
        Racionálne číslo obsahuje špeciálnu dvojicu bitov 01 na začiatku reťazca (na najvyžšich
        pozíciach). Na začiatku sekvencie zodpovedajúcej vykonániu stavu $q \in Q$ pribudne
        sekvencia, ktorá skontroluje, či sa aktuálna dvojica bitov pred desatinnou čiarkou nerovná
        špeciálnemu symbolu reprezentujúcemu prepadnutie hlavy. Ak áno, potom program končí hodnotou
        0. Inak pokračuje vykonávanie stavu.


\subsection*{B - RationalC na TS}
Racionálne číslo $x_0$ môžeme reprezentovať ako binárne kódovaný číslo. Na počiatku má teda páska
tvar $\Delta w \Delta$, kde w je binárne zakódované racionálne číslo a zároveň obsahuje TS sekvenciu
prechodov do prava, ktoré posunú hlavu na tú pozíciu, ktorá v pôvodnom racionálnom čísle
predstavovala desatinnú čiarku. Desatinná čiarka simulovaného registra $x_0$ sa v TS nachádza vždy
za pozíciou hlavy.
Prechod TS s posunutím hlavy doprava realizuje príkaz $x *= 2$, doľava zas $x /= 2$. Príkaz odd /
even sú realizované ako zápis symbolu 1 / 0 pod hlavou.
Každý riadok s príkazom programu RationalC je realizovaný ako jeden unikátny stav v TS.
Príkazy $x *= 2 / x /= 2$ sú prechody z aktuálneho stavu do stavu a posunutím hlavy,
príkazy even() či odd() zas zápisom symbolu pod hlavou na páske TS.
Príkaz return 1; predstavuje prechod pod ľubovoľným symbolom $\Sigma$ do koncového stavu TS,
príkaz return 0 zas spustenie podprogramu, ktorý bude nekonečne cykliť do prava.
Pre všetky stavy je pod symbolom \Delta pridaný podprogram, ktorý skočí na prvý \Delta na pravo od
hlavy, vyvolá shift right, skočí na prvý delta na ľavo a zapíše do neho symbol 0.
Týmto spôsobom je ošetrená situácia, kedy sa racionálne číslo delením rozšíruje smerom do ľava
(nulami), čo by spôsobilo prepadnutie hlavy v TS. 
Tak isto, pri násobení čísla tento podprogram korektne ošetrí rozšírenie binárneho kódovania o 0
zprava.

\end{document}
