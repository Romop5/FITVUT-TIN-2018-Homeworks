 
\documentclass[10pt]{article}

\usepackage[slovak]{babel}
\usepackage[utf8]{inputenc}
\usepackage[T1]{fontenc}
\usepackage{graphicx}

\usepackage{times}


\usepackage[margin=1in]{geometry} 
\usepackage{amsmath,amsthm,amssymb}
 
\newcommand{\N}{\mathbb{N}}
\newcommand{\Z}{\mathbb{Z}}
 
\newenvironment{theorem}[2][Theorem]{\begin{trivlist}
\item[\hskip \labelsep {\bfseries #1}\hskip \labelsep {\bfseries #2.}]}{\end{trivlist}}
\newenvironment{lemma}[2][Lemma]{\begin{trivlist}
\item[\hskip \labelsep {\bfseries #1}\hskip \labelsep {\bfseries #2.}]}{\end{trivlist}}
\newenvironment{exercise}[2][Exercise]{\begin{trivlist}
\item[\hskip \labelsep {\bfseries #1}\hskip \labelsep {\bfseries #2.}]}{\end{trivlist}}
\newenvironment{reflection}[2][Reflection]{\begin{trivlist}
\item[\hskip \labelsep {\bfseries #1}\hskip \labelsep {\bfseries #2.}]}{\end{trivlist}}
\newenvironment{proposition}[2][Proposition]{\begin{trivlist}
\item[\hskip \labelsep {\bfseries #1}\hskip \labelsep {\bfseries #2.}]}{\end{trivlist}}
\newenvironment{corollary}[2][Corollary]{\begin{trivlist}
\item[\hskip \labelsep {\bfseries #1}\hskip \labelsep {\bfseries #2.}]}{\end{trivlist}}
 
\begin{document}
 
% --------------------------------------------------------------
%                         Start here
% --------------------------------------------------------------
 
%\renewcommand{\qedsymbol}{\filledbox}
 
\title{TIN - Domáca úloha č. 3}%replace X with the appropriate number
\author{Roman Dobiáš - xdobia11@stud.fit.vutbr.cz}
 
\maketitle

\section*{Úloha č.1}
\section*{Úloha č.2}
\section*{Úloha č.3}
\section*{Úloha č.4}
Rozhodovací problém farbenia grafov je jazyk ColorGraph = $\{$ (<V>,<E>#k) | G = (<V>, <E>) je graf
ofarbitelný k farbami \}.
Redukcia z farbenia grafov na problem tedy Kvety 
\subsection*{Algoritmus prevodu}
Každú inštanciu jazyka ColorGraph sme schopný previesť na problém Tety Kvety nasledujúci:
\begin{itemize}
    \item Pre každý uzol E vygenerujeme K+1 surovín, kde každá surovina má kapacitu 1 (teda, Teda
        Kveta má práve 1 túto surovinu). K surovín reprezentuje jednotlivé z K farieb a K+1 surovina
        je použitá pre detekovanie, či už je vrchol ofarbený.
        Jednotlivé z K+1 surovín označme ako $E_i, 0 \leq i < k$.
    \item Pre každé ofarbenie uzla E farbou F 
    \begin{itemize}
        \item vypočítame množinu vrcholov I takých, že existuje hrana medzi vrcholom E a vrcholom z
            I a prizjednotíme vrchoľ E
        \item vytvoríme "pečivo" $E_F$, ktorého ingrediencie sú suroviny $a_i, a \in I, i = F$ a
            surovina $E_F$.
    \end{itemize}
    \item Pre takto zakódovaný problém riešime problém Tety Kvety pre počet priateľok $k$, kde $k = |V|$.
    \item Graf je ofarbiteľný práve vtedy, ak môžeme každý vrchol ofarbiť farbou tak, že
        priliehajúce vrcholy nemaju tú istú farbu.
    \item Zrejme platí, že ak upečiem pečivo $E_F$, potom toto pečivo bude pre daný vrchol jediné
        (vďaka surovine K+1) a zároveň priliehajúce vrcholy nebudú mať rovnaké pečivo (farbu),
        pretože surovina ich farby už bola vyčerpaná pri pečení $E_F$.
    \item Teda platí, že ak je možné upiecť N rôznych pečív, kde N je počet vrcholov a zároveň
        platia tézy vyššie, potom graf je K-ofarbiteľný.
\end{itemize}
\subsection*{Príklad}
Uvažujme graf A-B, B-C, C-D, D-A. 
Reprezentáciu úlohy môžeme vyjadriť tabulkou:
TODO
suroviny ako hlavička, pečio ako riadky, Y osa ako vybrané pečivá
\section*{Úloha č.5}
\end{document}
